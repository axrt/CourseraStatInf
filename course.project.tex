\documentclass[]{article}
\usepackage{lmodern}
\usepackage{amssymb,amsmath}
\usepackage{ifxetex,ifluatex}
\usepackage{fixltx2e} % provides \textsubscript
\ifnum 0\ifxetex 1\fi\ifluatex 1\fi=0 % if pdftex
  \usepackage[T1]{fontenc}
  \usepackage[utf8]{inputenc}
\else % if luatex or xelatex
  \ifxetex
    \usepackage{mathspec}
    \usepackage{xltxtra,xunicode}
  \else
    \usepackage{fontspec}
  \fi
  \defaultfontfeatures{Mapping=tex-text,Scale=MatchLowercase}
  \newcommand{\euro}{€}
\fi
% use upquote if available, for straight quotes in verbatim environments
\IfFileExists{upquote.sty}{\usepackage{upquote}}{}
% use microtype if available
\IfFileExists{microtype.sty}{%
\usepackage{microtype}
\UseMicrotypeSet[protrusion]{basicmath} % disable protrusion for tt fonts
}{}
\usepackage[margin=1in]{geometry}
\ifxetex
  \usepackage[setpagesize=false, % page size defined by xetex
              unicode=false, % unicode breaks when used with xetex
              xetex]{hyperref}
\else
  \usepackage[unicode=true]{hyperref}
\fi
\hypersetup{breaklinks=true,
            bookmarks=true,
            pdfauthor={Alexander Tuzhikov},
            pdftitle={Statistical Inference: The Study of the Exponential Distribution, A Simulation Exercise},
            colorlinks=true,
            citecolor=blue,
            urlcolor=blue,
            linkcolor=magenta,
            pdfborder={0 0 0}}
\urlstyle{same}  % don't use monospace font for urls
\usepackage{color}
\usepackage{fancyvrb}
\newcommand{\VerbBar}{|}
\newcommand{\VERB}{\Verb[commandchars=\\\{\}]}
\DefineVerbatimEnvironment{Highlighting}{Verbatim}{commandchars=\\\{\}}
% Add ',fontsize=\small' for more characters per line
\newenvironment{Shaded}{}{}
\newcommand{\KeywordTok}[1]{\textcolor[rgb]{0.00,0.00,1.00}{{#1}}}
\newcommand{\DataTypeTok}[1]{{#1}}
\newcommand{\DecValTok}[1]{{#1}}
\newcommand{\BaseNTok}[1]{{#1}}
\newcommand{\FloatTok}[1]{{#1}}
\newcommand{\ConstantTok}[1]{{#1}}
\newcommand{\CharTok}[1]{\textcolor[rgb]{0.00,0.50,0.50}{{#1}}}
\newcommand{\SpecialCharTok}[1]{\textcolor[rgb]{0.00,0.50,0.50}{{#1}}}
\newcommand{\StringTok}[1]{\textcolor[rgb]{0.00,0.50,0.50}{{#1}}}
\newcommand{\VerbatimStringTok}[1]{\textcolor[rgb]{0.00,0.50,0.50}{{#1}}}
\newcommand{\SpecialStringTok}[1]{\textcolor[rgb]{0.00,0.50,0.50}{{#1}}}
\newcommand{\ImportTok}[1]{{#1}}
\newcommand{\CommentTok}[1]{\textcolor[rgb]{0.00,0.50,0.00}{{#1}}}
\newcommand{\DocumentationTok}[1]{\textcolor[rgb]{0.00,0.50,0.00}{{#1}}}
\newcommand{\AnnotationTok}[1]{\textcolor[rgb]{0.00,0.50,0.00}{{#1}}}
\newcommand{\CommentVarTok}[1]{\textcolor[rgb]{0.00,0.50,0.00}{{#1}}}
\newcommand{\OtherTok}[1]{\textcolor[rgb]{1.00,0.25,0.00}{{#1}}}
\newcommand{\FunctionTok}[1]{{#1}}
\newcommand{\VariableTok}[1]{{#1}}
\newcommand{\ControlFlowTok}[1]{\textcolor[rgb]{0.00,0.00,1.00}{{#1}}}
\newcommand{\OperatorTok}[1]{{#1}}
\newcommand{\BuiltInTok}[1]{{#1}}
\newcommand{\ExtensionTok}[1]{{#1}}
\newcommand{\PreprocessorTok}[1]{\textcolor[rgb]{1.00,0.25,0.00}{{#1}}}
\newcommand{\AttributeTok}[1]{{#1}}
\newcommand{\RegionMarkerTok}[1]{{#1}}
\newcommand{\InformationTok}[1]{\textcolor[rgb]{0.00,0.50,0.00}{{#1}}}
\newcommand{\WarningTok}[1]{\textcolor[rgb]{0.00,0.50,0.00}{\textbf{{#1}}}}
\newcommand{\AlertTok}[1]{\textcolor[rgb]{1.00,0.00,0.00}{{#1}}}
\newcommand{\ErrorTok}[1]{\textcolor[rgb]{1.00,0.00,0.00}{\textbf{{#1}}}}
\newcommand{\NormalTok}[1]{{#1}}
\usepackage{graphicx,grffile}
\makeatletter
\def\maxwidth{\ifdim\Gin@nat@width>\linewidth\linewidth\else\Gin@nat@width\fi}
\def\maxheight{\ifdim\Gin@nat@height>\textheight\textheight\else\Gin@nat@height\fi}
\makeatother
% Scale images if necessary, so that they will not overflow the page
% margins by default, and it is still possible to overwrite the defaults
% using explicit options in \includegraphics[width, height, ...]{}
\setkeys{Gin}{width=\maxwidth,height=\maxheight,keepaspectratio}
\setlength{\parindent}{0pt}
\setlength{\parskip}{6pt plus 2pt minus 1pt}
\setlength{\emergencystretch}{3em}  % prevent overfull lines
\providecommand{\tightlist}{%
  \setlength{\itemsep}{0pt}\setlength{\parskip}{0pt}}
\setcounter{secnumdepth}{5}

%%% Use protect on footnotes to avoid problems with footnotes in titles
\let\rmarkdownfootnote\footnote%
\def\footnote{\protect\rmarkdownfootnote}

%%% Change title format to be more compact
\usepackage{titling}

% Create subtitle command for use in maketitle
\newcommand{\subtitle}[1]{
  \posttitle{
    \begin{center}\large#1\end{center}
    }
}

\setlength{\droptitle}{-2em}
  \title{Statistical Inference: The Study of the Exponential Distribution, A
Simulation Exercise}
  \pretitle{\vspace{\droptitle}\centering\huge}
  \posttitle{\par}
  \author{Alexander Tuzhikov}
  \preauthor{\centering\large\emph}
  \postauthor{\par}
  \predate{\centering\large\emph}
  \postdate{\par}
  \date{September 14, 2015}


% Redefines (sub)paragraphs to behave more like sections
\ifx\paragraph\undefined\else
\let\oldparagraph\paragraph
\renewcommand{\paragraph}[1]{\oldparagraph{#1}\mbox{}}
\fi
\ifx\subparagraph\undefined\else
\let\oldsubparagraph\subparagraph
\renewcommand{\subparagraph}[1]{\oldsubparagraph{#1}\mbox{}}
\fi

\begin{document}
\maketitle

{
\hypersetup{linkcolor=black}
\setcounter{tocdepth}{2}
\tableofcontents
}
\section{Overview: Exponential
Distribution}\label{overview-exponential-distribution}

In accordance with
\href{https://en.wikipedia.org/wiki/Exponential_distribution}{Wikipedia},
\textbf{e}xponential \textbf{d}istribution (ED) \emph{is the probability
distribution that describes the time between events in a Poisson
process, i.e.~a process in which events occur continuously and
independently at a constant average rate.} Both \emph{mean} and
\emph{standard deviation} of the ED is \emph{1/lambda}. As suggested in
the study objective, here we will use lambda = 0.2. However, for the
purpose of introduction, let's reconstruct the wiki plots of the ED with
different lambda:

\begin{Shaded}
\begin{Highlighting}[]
\NormalTok{lambdas<-}\StringTok{ }\KeywordTok{c}\NormalTok{(}\FloatTok{0.2}\NormalTok{, }\FloatTok{0.5}\NormalTok{, }\DecValTok{1}\NormalTok{, }\FloatTok{1.5}\NormalTok{)}\CommentTok{#the given in the task + those from wikipedia}
\NormalTok{n<-}\StringTok{ }\DecValTok{40} \CommentTok{#given by ".. you will investigate the distribution of }
       \CommentTok{#averages of 40 exponential(0.2)s" in the task}
\NormalTok{sampling.count<-}\StringTok{ }\DecValTok{1000}  \CommentTok{#given by ".. you will need to do a }
       \CommentTok{#thousand or so simulated averages of 40 exponentials" in the task}
\CommentTok{#prepare a data.frame for the plot, melt by x, plot as line}
\NormalTok{ed.plot.df<-}\StringTok{ }\KeywordTok{as.data.frame}\NormalTok{(}\KeywordTok{cbind}\NormalTok{(}
        \DataTypeTok{x=}\DecValTok{0}\NormalTok{:}\DecValTok{40}\NormalTok{,}
        \DataTypeTok{la.0.2=}\KeywordTok{dexp}\NormalTok{(}\DataTypeTok{x=}\DecValTok{0}\NormalTok{:}\DecValTok{40}\NormalTok{, lambdas[}\DecValTok{1}\NormalTok{]),}
        \DataTypeTok{la.0.5=}\KeywordTok{dexp}\NormalTok{(}\DataTypeTok{x=}\DecValTok{0}\NormalTok{:}\DecValTok{40}\NormalTok{, lambdas[}\DecValTok{2}\NormalTok{]),}
        \DataTypeTok{la.1=}\KeywordTok{dexp}\NormalTok{(}\DataTypeTok{x=}\DecValTok{0}\NormalTok{:}\DecValTok{40}\NormalTok{, lambdas[}\DecValTok{3}\NormalTok{]),}
        \DataTypeTok{la.1.5=}\KeywordTok{dexp}\NormalTok{(}\DataTypeTok{x=}\DecValTok{0}\NormalTok{:}\DecValTok{40}\NormalTok{, lambdas[}\DecValTok{4}\NormalTok{])}
\NormalTok{)) %>%}\StringTok{ }
\StringTok{        }\KeywordTok{melt}\NormalTok{(}\DataTypeTok{id.vars=}\StringTok{"x"}\NormalTok{) %>%}
\StringTok{        }\KeywordTok{ggplot}\NormalTok{(}\DataTypeTok{data=}\NormalTok{., }\DataTypeTok{mapping=}\KeywordTok{aes}\NormalTok{(}\DataTypeTok{x=}\NormalTok{x, }\DataTypeTok{group=}\NormalTok{variable, }\DataTypeTok{y=}\NormalTok{value, }\DataTypeTok{color=}\NormalTok{variable)) +}\StringTok{ }
\StringTok{        }\KeywordTok{geom_line}\NormalTok{() +}\StringTok{ }\KeywordTok{theme_bw}\NormalTok{() +}\StringTok{ }\KeywordTok{xlim}\NormalTok{(}\DecValTok{0}\NormalTok{,}\DecValTok{20}\NormalTok{) +}\StringTok{ }\KeywordTok{ylim}\NormalTok{(}\DecValTok{0}\NormalTok{,}\DecValTok{1}\NormalTok{) +}
\StringTok{        }\KeywordTok{labs}\NormalTok{(}\DataTypeTok{title=}\StringTok{"ED Probability Density Function"}\NormalTok{) +}\StringTok{ }\KeywordTok{ylab}\NormalTok{(}\StringTok{"Probability"}\NormalTok{) +}
\StringTok{        }\KeywordTok{scale_color_manual}\NormalTok{(}\DataTypeTok{values=}\KeywordTok{rainbow}\NormalTok{(}\DecValTok{4}\NormalTok{, }\DataTypeTok{start =} \FloatTok{0.4}\NormalTok{, }\DataTypeTok{end =} \FloatTok{0.8}\NormalTok{), }\DataTypeTok{labels=}\KeywordTok{c}\NormalTok{(}\StringTok{"0.2"}\NormalTok{,}\StringTok{"0.5"}\NormalTok{, }\StringTok{"1.0"}\NormalTok{, }\StringTok{"1.5"}\NormalTok{), }\DataTypeTok{name=}\StringTok{"Lambda"}\NormalTok{)}
\KeywordTok{plot}\NormalTok{(ed.plot.df)}
\end{Highlighting}
\end{Shaded}

\begin{verbatim}
## Warning: Removed 81 rows containing missing values (geom_path).
\end{verbatim}

\includegraphics{course.project_files/figure-latex/Overview 1-1.pdf}

\section{Simulations}\label{simulations}

\section{Sample Mean versus Theoretical
Mean}\label{sample-mean-versus-theoretical-mean}

\section{Sample Variance versus Theoretical
Variance}\label{sample-variance-versus-theoretical-variance}

\section{Distribution}\label{distribution}

\end{document}
