\documentclass[]{article}
\usepackage{lmodern}
\usepackage{amssymb,amsmath}
\usepackage{ifxetex,ifluatex}
\usepackage{fixltx2e} % provides \textsubscript
\ifnum 0\ifxetex 1\fi\ifluatex 1\fi=0 % if pdftex
  \usepackage[T1]{fontenc}
  \usepackage[utf8]{inputenc}
\else % if luatex or xelatex
  \ifxetex
    \usepackage{mathspec}
    \usepackage{xltxtra,xunicode}
  \else
    \usepackage{fontspec}
  \fi
  \defaultfontfeatures{Mapping=tex-text,Scale=MatchLowercase}
  \newcommand{\euro}{€}
    \setmainfont{Ubuntu}
\fi
% use upquote if available, for straight quotes in verbatim environments
\IfFileExists{upquote.sty}{\usepackage{upquote}}{}
% use microtype if available
\IfFileExists{microtype.sty}{%
\usepackage{microtype}
\UseMicrotypeSet[protrusion]{basicmath} % disable protrusion for tt fonts
}{}
\usepackage[margin=1in]{geometry}
\ifxetex
  \usepackage[setpagesize=false, % page size defined by xetex
              unicode=false, % unicode breaks when used with xetex
              xetex]{hyperref}
\else
  \usepackage[unicode=true]{hyperref}
\fi
\hypersetup{breaklinks=true,
            bookmarks=true,
            pdfauthor={Alexander Tuzhikov},
            pdftitle={Statistical Inference: The Study of the Exponential Distribution, A Simulation Exercise},
            colorlinks=true,
            citecolor=blue,
            urlcolor=blue,
            linkcolor=magenta,
            pdfborder={0 0 0}}
\urlstyle{same}  % don't use monospace font for urls
\usepackage{color}
\usepackage{fancyvrb}
\newcommand{\VerbBar}{|}
\newcommand{\VERB}{\Verb[commandchars=\\\{\}]}
\DefineVerbatimEnvironment{Highlighting}{Verbatim}{commandchars=\\\{\}}
% Add ',fontsize=\small' for more characters per line
\newenvironment{Shaded}{}{}
\newcommand{\KeywordTok}[1]{\textcolor[rgb]{0.00,0.00,1.00}{{#1}}}
\newcommand{\DataTypeTok}[1]{{#1}}
\newcommand{\DecValTok}[1]{{#1}}
\newcommand{\BaseNTok}[1]{{#1}}
\newcommand{\FloatTok}[1]{{#1}}
\newcommand{\ConstantTok}[1]{{#1}}
\newcommand{\CharTok}[1]{\textcolor[rgb]{0.00,0.50,0.50}{{#1}}}
\newcommand{\SpecialCharTok}[1]{\textcolor[rgb]{0.00,0.50,0.50}{{#1}}}
\newcommand{\StringTok}[1]{\textcolor[rgb]{0.00,0.50,0.50}{{#1}}}
\newcommand{\VerbatimStringTok}[1]{\textcolor[rgb]{0.00,0.50,0.50}{{#1}}}
\newcommand{\SpecialStringTok}[1]{\textcolor[rgb]{0.00,0.50,0.50}{{#1}}}
\newcommand{\ImportTok}[1]{{#1}}
\newcommand{\CommentTok}[1]{\textcolor[rgb]{0.00,0.50,0.00}{{#1}}}
\newcommand{\DocumentationTok}[1]{\textcolor[rgb]{0.00,0.50,0.00}{{#1}}}
\newcommand{\AnnotationTok}[1]{\textcolor[rgb]{0.00,0.50,0.00}{{#1}}}
\newcommand{\CommentVarTok}[1]{\textcolor[rgb]{0.00,0.50,0.00}{{#1}}}
\newcommand{\OtherTok}[1]{\textcolor[rgb]{1.00,0.25,0.00}{{#1}}}
\newcommand{\FunctionTok}[1]{{#1}}
\newcommand{\VariableTok}[1]{{#1}}
\newcommand{\ControlFlowTok}[1]{\textcolor[rgb]{0.00,0.00,1.00}{{#1}}}
\newcommand{\OperatorTok}[1]{{#1}}
\newcommand{\BuiltInTok}[1]{{#1}}
\newcommand{\ExtensionTok}[1]{{#1}}
\newcommand{\PreprocessorTok}[1]{\textcolor[rgb]{1.00,0.25,0.00}{{#1}}}
\newcommand{\AttributeTok}[1]{{#1}}
\newcommand{\RegionMarkerTok}[1]{{#1}}
\newcommand{\InformationTok}[1]{\textcolor[rgb]{0.00,0.50,0.00}{{#1}}}
\newcommand{\WarningTok}[1]{\textcolor[rgb]{0.00,0.50,0.00}{\textbf{{#1}}}}
\newcommand{\AlertTok}[1]{\textcolor[rgb]{1.00,0.00,0.00}{{#1}}}
\newcommand{\ErrorTok}[1]{\textcolor[rgb]{1.00,0.00,0.00}{\textbf{{#1}}}}
\newcommand{\NormalTok}[1]{{#1}}
\usepackage{graphicx,grffile}
\makeatletter
\def\maxwidth{\ifdim\Gin@nat@width>\linewidth\linewidth\else\Gin@nat@width\fi}
\def\maxheight{\ifdim\Gin@nat@height>\textheight\textheight\else\Gin@nat@height\fi}
\makeatother
% Scale images if necessary, so that they will not overflow the page
% margins by default, and it is still possible to overwrite the defaults
% using explicit options in \includegraphics[width, height, ...]{}
\setkeys{Gin}{width=\maxwidth,height=\maxheight,keepaspectratio}
\setlength{\parindent}{0pt}
\setlength{\parskip}{6pt plus 2pt minus 1pt}
\setlength{\emergencystretch}{3em}  % prevent overfull lines
\providecommand{\tightlist}{%
  \setlength{\itemsep}{0pt}\setlength{\parskip}{0pt}}
\setcounter{secnumdepth}{5}

%%% Use protect on footnotes to avoid problems with footnotes in titles
\let\rmarkdownfootnote\footnote%
\def\footnote{\protect\rmarkdownfootnote}

%%% Change title format to be more compact
\usepackage{titling}

% Create subtitle command for use in maketitle
\newcommand{\subtitle}[1]{
  \posttitle{
    \begin{center}\large#1\end{center}
    }
}

\setlength{\droptitle}{-2em}
  \title{Statistical Inference: The Study of the Exponential Distribution, A
Simulation Exercise}
  \pretitle{\vspace{\droptitle}\centering\huge}
  \posttitle{\par}
  \author{Alexander Tuzhikov}
  \preauthor{\centering\large\emph}
  \postauthor{\par}
  \predate{\centering\large\emph}
  \postdate{\par}
  \date{September 14, 2015}


% Redefines (sub)paragraphs to behave more like sections
\ifx\paragraph\undefined\else
\let\oldparagraph\paragraph
\renewcommand{\paragraph}[1]{\oldparagraph{#1}\mbox{}}
\fi
\ifx\subparagraph\undefined\else
\let\oldsubparagraph\subparagraph
\renewcommand{\subparagraph}[1]{\oldsubparagraph{#1}\mbox{}}
\fi

\begin{document}
\maketitle

\section{Overview: Exponential
Distribution}\label{overview-exponential-distribution}

In accordance with
\href{https://en.wikipedia.org/wiki/Exponential_distribution}{Wikipedia},
\textbf{e}xponential \textbf{d}istribution (ED) \emph{is the probability
distribution that describes the time between events in a Poisson
process, i.e.~a process in which events occur continuously and
independently at a constant average rate.} Both \emph{mean} and
\emph{standard deviation} of the ED is \emph{1/lambda}. As suggested in
the study objective, here we will use \textbf{lambda = 0.2}. However,
for the purpose of introduction, let's reconstruct the wiki plots of the
ED with different lambda:

\begin{center}\includegraphics{course.project_files/figure-latex/Overview 3-1} \end{center}

It is obvious that ED is a skewed distribution, which drastically
differes from the standard normal bell-shaped curve. We can always
double check if the meand and standard deviation are indded equal in our
example (see \textbf{\hyperref[code-block-1]{Code Block 1}}, for the
plot see \textbf{\hyperref[code-block-2]{Code Block 2}}).

\section{Simulations: Sampling the
ED}\label{simulations-sampling-the-ed}

Now, let's move to the first task:\\
\emph{1. Show the sample mean and compare it to the theoretical mean of
the distribution.} First we gonna need 1000 samples of size 40 form ED.
We will generate a matrix of 40 rows by 1000 columns, calculate the
means and store them in a vector for further reuse (see
\textbf{\hyperref[code-block-3]{Code Block 3}})

\section{Sample Mean versus Theoretical
Mean}\label{sample-mean-versus-theoretical-mean}

Ok, now let's compare the theoretical mean, which is \textbf{1/lambda},
to that of the simulation procedure:

\begin{Shaded}
\begin{Highlighting}[]
\NormalTok{ed.mean.theo<-}\StringTok{ }\DecValTok{1}\NormalTok{/lambdas[}\DecValTok{1}\NormalTok{]}
\NormalTok{ed.mean.sim<-}\StringTok{ }\KeywordTok{mean}\NormalTok{(samples.colMeans) }\CommentTok{#calculate the total mean}
\KeywordTok{print}\NormalTok{(}\KeywordTok{c}\NormalTok{(}\DataTypeTok{theoretical.mean=} \NormalTok{ed.mean.theo, }\DataTypeTok{simulation.mean=}\NormalTok{ed.mean.sim))}
\end{Highlighting}
\end{Shaded}

\begin{verbatim}
## theoretical.mean  simulation.mean 
##         5.000000         5.016687
\end{verbatim}

The difference is neglegible in our case.

\section{Sample Variance versus Theoretical
Variance}\label{sample-variance-versus-theoretical-variance}

Now we will do the same in order to calculate the variance and see if it
differs from the theoretical variance, as being asked in the second
task: \emph{2. Show how variable it is and compare it to the theoretical
variance of the distribution.}

\begin{Shaded}
\begin{Highlighting}[]
\NormalTok{ed.var.sim<-}\StringTok{ }\KeywordTok{var}\NormalTok{(}\KeywordTok{colMeans}\NormalTok{(samples))}\CommentTok{#simulated variance}
\NormalTok{ed.var.theo<-}\StringTok{ }\NormalTok{(}\DecValTok{1}\NormalTok{/lambdas[}\DecValTok{1}\NormalTok{]/}\KeywordTok{sqrt}\NormalTok{(n))^}\DecValTok{2}\CommentTok{#the theoretical variance is}
\KeywordTok{print}\NormalTok{(}\KeywordTok{c}\NormalTok{(}\DataTypeTok{thoretical.var=}\NormalTok{ed.var.theo, }\DataTypeTok{simulated.var=}\NormalTok{ed.var.sim))}
\end{Highlighting}
\end{Shaded}

\begin{verbatim}
## thoretical.var  simulated.var 
##       0.625000       0.605856
\end{verbatim}

\section{Distribution: Sampling Means Are Distributed Approximately
Normaly}\label{distribution-sampling-means-are-distributed-approximately-normaly}

Now we move to the third task: \emph{3. Show that the distribution is
approximately normal.}

\includegraphics{course.project_files/figure-latex/distribution 1-1.pdf}
The above two plots demonstrate that the simulated meand are very close
to be distributed normally. Finally, let's see how \emph{the
distribution of a large collection of random exponentials and the
distribution of a large collection of averages of 40 exponentials}
differ (see \textbf{\hyperref[code-block-4]{Code Block 4}})

\includegraphics{course.project_files/figure-latex/distribution 2-1.pdf}
The above plot shows how the sampling means becomes more and more
``Normal distribution''-like shaped with the increase in the number of
samples taken, while the density of ED becomes strongly non-normal (see
\textbf{\hyperref[code-block-5]{Code Block 5}}).

\section{Related R Code}\label{related-r-code}

\subsection{Code Block 0}\label{code-block-0}

\begin{Shaded}
\begin{Highlighting}[]
\NormalTok{### Code block 1: libraries}
\KeywordTok{library}\NormalTok{(dplyr)}
\KeywordTok{library}\NormalTok{(ggplot2)}
\KeywordTok{library}\NormalTok{(reshape2)}
\end{Highlighting}
\end{Shaded}

\hyperdef{}{code-block-1}{\subsection{Code Block 1}\label{code-block-1}}

\begin{Shaded}
\begin{Highlighting}[]
\NormalTok{ed.mean<-}\StringTok{ }\KeywordTok{mean}\NormalTok{(}\KeywordTok{rexp}\NormalTok{(}\FloatTok{1e6}\NormalTok{, }\FloatTok{0.2}\NormalTok{)) }\CommentTok{#generate values from ED}
\NormalTok{ed.sd<-}\StringTok{ }\KeywordTok{sd}\NormalTok{(}\KeywordTok{rexp}\NormalTok{(}\FloatTok{1e6}\NormalTok{, }\FloatTok{0.2}\NormalTok{))}
\KeywordTok{all.equal}\NormalTok{(ed.mean, ed.sd, }\DataTypeTok{tolerance =} \FloatTok{1e-2}\NormalTok{) }\CommentTok{#equal up to 1e-2 level of prescision}
\end{Highlighting}
\end{Shaded}

\hyperdef{}{code-block-2}{\subsection{Code Block 2}\label{code-block-2}}

\begin{Shaded}
\begin{Highlighting}[]
\NormalTok{### Code block 2: ED with different lambdas}
\CommentTok{#prepare a data.frame for the plot, melt by x, plot as line}
\NormalTok{ed.plot.df<-}\StringTok{ }\KeywordTok{as.data.frame}\NormalTok{(}\KeywordTok{cbind}\NormalTok{(}
        \DataTypeTok{x=}\DecValTok{0}\NormalTok{:}\DecValTok{40}\NormalTok{,}
        \DataTypeTok{la.0.2=}\KeywordTok{dexp}\NormalTok{(}\DataTypeTok{x=}\DecValTok{0}\NormalTok{:}\DecValTok{40}\NormalTok{, lambdas[}\DecValTok{1}\NormalTok{]),}
        \DataTypeTok{la.0.5=}\KeywordTok{dexp}\NormalTok{(}\DataTypeTok{x=}\DecValTok{0}\NormalTok{:}\DecValTok{40}\NormalTok{, lambdas[}\DecValTok{2}\NormalTok{]),}
        \DataTypeTok{la.1=}\KeywordTok{dexp}\NormalTok{(}\DataTypeTok{x=}\DecValTok{0}\NormalTok{:}\DecValTok{40}\NormalTok{, lambdas[}\DecValTok{3}\NormalTok{]),}
        \DataTypeTok{la.1.5=}\KeywordTok{dexp}\NormalTok{(}\DataTypeTok{x=}\DecValTok{0}\NormalTok{:}\DecValTok{40}\NormalTok{, lambdas[}\DecValTok{4}\NormalTok{])}
\NormalTok{)) %>%}\StringTok{ }
\StringTok{        }\KeywordTok{melt}\NormalTok{(}\DataTypeTok{id.vars=}\StringTok{"x"}\NormalTok{) %>%}
\StringTok{        }\KeywordTok{ggplot}\NormalTok{(}\DataTypeTok{data=}\NormalTok{., }\DataTypeTok{mapping=}\KeywordTok{aes}\NormalTok{(}\DataTypeTok{x=}\NormalTok{x, }\DataTypeTok{group=}\NormalTok{variable, }\DataTypeTok{y=}\NormalTok{value, }\DataTypeTok{color=}\NormalTok{variable)) +}\StringTok{ }
\StringTok{        }\KeywordTok{geom_line}\NormalTok{(}\DataTypeTok{size=}\DecValTok{1}\NormalTok{) +}\StringTok{ }\KeywordTok{theme_bw}\NormalTok{() +}\StringTok{ }\KeywordTok{xlim}\NormalTok{(}\DecValTok{0}\NormalTok{,}\DecValTok{20}\NormalTok{) +}\StringTok{ }\KeywordTok{ylim}\NormalTok{(}\DecValTok{0}\NormalTok{,}\DecValTok{1}\NormalTok{) +}
\StringTok{        }\KeywordTok{labs}\NormalTok{(}\DataTypeTok{title=}\StringTok{"ED Probability Density Function"}\NormalTok{) +}\StringTok{ }\KeywordTok{ylab}\NormalTok{(}\StringTok{"Probability"}\NormalTok{) +}
\StringTok{        }\KeywordTok{scale_color_manual}\NormalTok{(}\DataTypeTok{values=}\KeywordTok{c}\NormalTok{(}\StringTok{"violetred"}\NormalTok{, }\StringTok{"turquoise3"}\NormalTok{, }\StringTok{"springgreen4"}\NormalTok{, }
                                    \StringTok{"salmon4"}\NormalTok{), }\DataTypeTok{labels=}\KeywordTok{c}\NormalTok{(}\StringTok{"0.2"}\NormalTok{,}\StringTok{"0.5"}\NormalTok{, }\StringTok{"1.0"}\NormalTok{, }\StringTok{"1.5"}\NormalTok{), }
                           \DataTypeTok{name=}\StringTok{"Lambda"}\NormalTok{)+}
\StringTok{        }\KeywordTok{theme}\NormalTok{(}\DataTypeTok{legend.key =} \KeywordTok{element_rect}\NormalTok{(}\DataTypeTok{colour =} \OtherTok{NA}\NormalTok{))}
\KeywordTok{plot}\NormalTok{(ed.plot.df)}
\end{Highlighting}
\end{Shaded}

\hyperdef{}{code-block-3}{\subsection{Code Block 3}\label{code-block-3}}

\begin{Shaded}
\begin{Highlighting}[]
\NormalTok{### Code block 3: sampling}
\CommentTok{#sampling}
\NormalTok{samples<-}\StringTok{ }\KeywordTok{as.data.frame}\NormalTok{(}\KeywordTok{do.call}\NormalTok{(}\DataTypeTok{what =} \StringTok{"cbind"}\NormalTok{, }
                                \DataTypeTok{args =} \KeywordTok{lapply}\NormalTok{(}\DecValTok{1}\NormalTok{:sampling.count, }
                                              \NormalTok{function(x)\{}\KeywordTok{return}\NormalTok{(}\KeywordTok{rexp}\NormalTok{(n, lambdas[}\DecValTok{1}\NormalTok{]))\})))}
\NormalTok{samples.colMeans<-}\StringTok{ }\KeywordTok{colMeans}\NormalTok{(samples) }\CommentTok{#calculate the column means}
\end{Highlighting}
\end{Shaded}

\hyperdef{}{code-block-4}{\subsection{Code Block 4}\label{code-block-4}}

\begin{Shaded}
\begin{Highlighting}[]
\CommentTok{#generate normally distributed data and combine both menas and the generated data}
\KeywordTok{as.data.frame}\NormalTok{(}\KeywordTok{cbind}\NormalTok{(}\DataTypeTok{n=}\DecValTok{1}\NormalTok{:sampling.count, }\DataTypeTok{sampling.means=}\NormalTok{samples.colMeans, }
                    \DataTypeTok{normal.data=}\KeywordTok{dnorm}\NormalTok{(}\KeywordTok{seq}\NormalTok{(}\FloatTok{0.01}\NormalTok{, }\DecValTok{10}\NormalTok{, }\FloatTok{0.01}\NormalTok{), }
                                      \DataTypeTok{mean =} \NormalTok{ed.mean.theo, }
                                      \DataTypeTok{sd =} \KeywordTok{sqrt}\NormalTok{(ed.var.theo)),}
                    \DataTypeTok{normal.prob=}\KeywordTok{rnorm}\NormalTok{(}\DecValTok{1000}\NormalTok{, }
                                      \DataTypeTok{mean =} \NormalTok{ed.mean.theo, }
                                      \DataTypeTok{sd =} \KeywordTok{sqrt}\NormalTok{(ed.var.theo)))) ->}\StringTok{ }\NormalTok{ed.plot.data}
\NormalTok{combined.theo.sim.plot<-}\StringTok{ }\KeywordTok{ggplot}\NormalTok{() +}\StringTok{  }
\StringTok{        }\KeywordTok{geom_histogram}\NormalTok{(}\DataTypeTok{data=}\NormalTok{ed.plot.data, }
                       \DataTypeTok{mapping=}\KeywordTok{aes}\NormalTok{(}\DataTypeTok{x=}\NormalTok{samples.colMeans, }
                                   \DataTypeTok{y=}\NormalTok{..density..,}
                                   \DataTypeTok{fill=}\StringTok{"lightsteelblue1"}\NormalTok{),}
                       \DataTypeTok{color=}\StringTok{"skyblue3"}\NormalTok{, }
                       \DataTypeTok{stat=}\StringTok{"bin"}\NormalTok{,}\DataTypeTok{binwidth=}\FloatTok{0.25}\NormalTok{)+}
\StringTok{        }\KeywordTok{geom_line}\NormalTok{(}\DataTypeTok{data=}\NormalTok{ed.plot.data, }\DataTypeTok{mapping=}\KeywordTok{aes}\NormalTok{(}\DataTypeTok{x=}\KeywordTok{seq}\NormalTok{(}\FloatTok{0.01}\NormalTok{, }\DecValTok{10}\NormalTok{, }\FloatTok{0.01}\NormalTok{),}
                                                 \DataTypeTok{y=} \NormalTok{normal.data,}
                                                 \DataTypeTok{color =} \StringTok{"chocolate"}\NormalTok{), }
                  \DataTypeTok{size=}\FloatTok{1.5}\NormalTok{) +}\StringTok{ }
\StringTok{        }\KeywordTok{geom_vline}\NormalTok{(}\KeywordTok{aes}\NormalTok{(}\DataTypeTok{xintercept=}\NormalTok{ed.mean.sim, }\DataTypeTok{color=}\StringTok{"green"}\NormalTok{)) +}
\StringTok{        }\KeywordTok{geom_vline}\NormalTok{(}\KeywordTok{aes}\NormalTok{(}\DataTypeTok{xintercept=}\NormalTok{ed.mean.theo, }\DataTypeTok{color=}\StringTok{"red"}\NormalTok{)) +}
\StringTok{        }\KeywordTok{xlim}\NormalTok{(}\FloatTok{2.5}\NormalTok{, }\FloatTok{7.5}\NormalTok{)+}\KeywordTok{theme_bw}\NormalTok{() +}\StringTok{ }
\StringTok{        }\KeywordTok{labs}\NormalTok{(}\DataTypeTok{title=}\StringTok{"Sample Distribution vs Theoretical Distribution"}\NormalTok{) +}\StringTok{ }
\StringTok{        }\KeywordTok{xlab}\NormalTok{(}\StringTok{"Means"}\NormalTok{) +}\StringTok{ }\KeywordTok{ylab}\NormalTok{(}\StringTok{"Density"}\NormalTok{) +}
\StringTok{        }\KeywordTok{scale_fill_identity}\NormalTok{(}\DataTypeTok{name =} \StringTok{""}\NormalTok{, }\DataTypeTok{guide =} \StringTok{"legend"}\NormalTok{,}
                            \DataTypeTok{labels =} \KeywordTok{c}\NormalTok{(}\StringTok{"Simulated Means"}\NormalTok{)) +}
\StringTok{        }\KeywordTok{scale_colour_manual}\NormalTok{(}\DataTypeTok{name =} \StringTok{""}\NormalTok{, }
                            \DataTypeTok{values =} \KeywordTok{c}\NormalTok{(}\StringTok{"chocolate"}\NormalTok{=}\StringTok{"chocolate"}\NormalTok{,}
                                       \StringTok{"red"}\NormalTok{=}\StringTok{"red"}\NormalTok{, }\StringTok{"green"}\NormalTok{=}\StringTok{"green"}\NormalTok{), }
                            \DataTypeTok{labels =} \KeywordTok{c}\NormalTok{(}\StringTok{"Theoretical Distribution"}\NormalTok{,}
                                       \StringTok{"Theoretical Mean"}\NormalTok{, }\StringTok{"Simulated Mean"}\NormalTok{)) +}
\StringTok{        }\KeywordTok{theme}\NormalTok{(}\DataTypeTok{legend.key =} \KeywordTok{element_rect}\NormalTok{(}\DataTypeTok{colour =} \OtherTok{NA}\NormalTok{), }\DataTypeTok{legend.position=}\StringTok{"bottom"}\NormalTok{,}\DataTypeTok{legend.box =} \StringTok{"horizontal"}\NormalTok{, }
              \DataTypeTok{legend.text=}\KeywordTok{element_text}\NormalTok{(}\DataTypeTok{size=} \DecValTok{7}\NormalTok{))}
\NormalTok{qq.plot<-}\StringTok{ }\KeywordTok{ggplot}\NormalTok{() +}\StringTok{ }\KeywordTok{stat_qq}\NormalTok{(}\DataTypeTok{data=}\NormalTok{ed.plot.data, }
                             \DataTypeTok{mapping=}\KeywordTok{aes}\NormalTok{(}\DataTypeTok{sample=}\NormalTok{sampling.means,}\DataTypeTok{color=}\StringTok{"skyblue3"}\NormalTok{)) +}
\StringTok{        }\KeywordTok{stat_qq}\NormalTok{(}\DataTypeTok{data=}\NormalTok{ed.plot.data, }
                \DataTypeTok{mapping=}\KeywordTok{aes}\NormalTok{(}\DataTypeTok{sample=}\NormalTok{normal.prob,}\DataTypeTok{color=}\StringTok{"chocolate"}\NormalTok{)) +}\StringTok{ }
\StringTok{        }\KeywordTok{theme_bw}\NormalTok{() +}
\StringTok{        }\KeywordTok{xlab}\NormalTok{(}\StringTok{"Theoretical"}\NormalTok{) +}\StringTok{ }\KeywordTok{ylab}\NormalTok{(}\StringTok{"Sample"}\NormalTok{) +}\StringTok{ }\KeywordTok{labs}\NormalTok{(}\DataTypeTok{title=}\StringTok{"Normal QQ-Plot"}\NormalTok{) +}\StringTok{ }
\StringTok{        }\KeywordTok{scale_color_manual}\NormalTok{(}\DataTypeTok{name=}\StringTok{""}\NormalTok{, }
                           \DataTypeTok{values=}\KeywordTok{c}\NormalTok{(}\StringTok{"skyblue3"}\NormalTok{=}\StringTok{"skyblue3"}\NormalTok{,}\StringTok{"chocolate"}\NormalTok{=}\StringTok{"chocolate"}\NormalTok{), }
                           \DataTypeTok{labels=}\KeywordTok{c}\NormalTok{(}\StringTok{"Sampling Means"}\NormalTok{,}\StringTok{"Normal Data"}\NormalTok{))+}
\StringTok{        }\KeywordTok{theme}\NormalTok{(}\DataTypeTok{legend.key =} \KeywordTok{element_rect}\NormalTok{(}\DataTypeTok{colour =} \OtherTok{NA}\NormalTok{), }\DataTypeTok{legend.position=}\StringTok{"bottom"}\NormalTok{,}
              \DataTypeTok{legend.text=}\KeywordTok{element_text}\NormalTok{(}\DataTypeTok{size=} \DecValTok{7}\NormalTok{))}

\KeywordTok{library}\NormalTok{(gridExtra)}

\KeywordTok{grid.arrange}\NormalTok{(combined.theo.sim.plot, qq.plot, }\DataTypeTok{ncol=}\DecValTok{2}\NormalTok{,}\DataTypeTok{widths=}\KeywordTok{c}\NormalTok{(}\DecValTok{2}\NormalTok{, }\DecValTok{1}\NormalTok{))}
\end{Highlighting}
\end{Shaded}

\hyperdef{}{code-block-5}{\subsection{Code Block 5}\label{code-block-5}}

\begin{Shaded}
\begin{Highlighting}[]
\NormalTok{ed.means.hist<-}\StringTok{ }\KeywordTok{ggplot}\NormalTok{() +}\StringTok{ }
\StringTok{        }\KeywordTok{geom_histogram}\NormalTok{(}\DataTypeTok{data=}\NormalTok{ed.plot.data, }
                       \DataTypeTok{mapping=}\KeywordTok{aes}\NormalTok{(}\DataTypeTok{x=}\NormalTok{sampling.means, }\DataTypeTok{y=}\NormalTok{..density..), }
                       \DataTypeTok{fill=}\StringTok{"lightsteelblue1"}\NormalTok{, }
                       \DataTypeTok{color=}\StringTok{"skyblue3"}\NormalTok{, }\DataTypeTok{stat=}\StringTok{"bin"}\NormalTok{, }\DataTypeTok{binwidth=}\FloatTok{0.25}\NormalTok{) +}
\StringTok{        }\KeywordTok{theme_bw}\NormalTok{() +}\StringTok{ }\KeywordTok{xlab}\NormalTok{(}\StringTok{"Sampling Means"}\NormalTok{) +}\StringTok{ }\KeywordTok{ylab}\NormalTok{(}\StringTok{"Density"}\NormalTok{) +}\StringTok{ }
\StringTok{        }\KeywordTok{labs}\NormalTok{(}\DataTypeTok{title=}\StringTok{"Sampling Means Density"}\NormalTok{) +}\StringTok{  }
\StringTok{        }\KeywordTok{xlim}\NormalTok{(}\FloatTok{2.5}\NormalTok{, }\FloatTok{7.5}\NormalTok{)+}
\StringTok{        }\KeywordTok{theme}\NormalTok{(}\DataTypeTok{legend.key =} \KeywordTok{element_rect}\NormalTok{(}\DataTypeTok{colour =} \OtherTok{NA}\NormalTok{))}
\NormalTok{ed.values.hist<-}\StringTok{ }\KeywordTok{ggplot}\NormalTok{() +}
\StringTok{        }\KeywordTok{geom_histogram}\NormalTok{(}\DataTypeTok{data=}\KeywordTok{melt}\NormalTok{(samples), }
                       \DataTypeTok{mapping=}\KeywordTok{aes}\NormalTok{(}\DataTypeTok{x=}\NormalTok{value, }\DataTypeTok{y=}\NormalTok{..density..), }
                       \DataTypeTok{fill=}\StringTok{"palegreen"}\NormalTok{, }
                       \DataTypeTok{color=}\StringTok{"darkgreen"}\NormalTok{, }
                       \DataTypeTok{stat=}\StringTok{"bin"}\NormalTok{, }
                       \DataTypeTok{binwidth=}\FloatTok{0.25}\NormalTok{) +}
\StringTok{        }\KeywordTok{theme_bw}\NormalTok{() +}\StringTok{ }\KeywordTok{xlab}\NormalTok{(}\StringTok{"ED Simulations"}\NormalTok{) +}\StringTok{ }\KeywordTok{ylab}\NormalTok{(}\StringTok{"Density"}\NormalTok{) +}\StringTok{ }
\StringTok{        }\KeywordTok{labs}\NormalTok{(}\DataTypeTok{title=}\StringTok{"ED Density"}\NormalTok{)+}\StringTok{  }\KeywordTok{xlim}\NormalTok{(}\DecValTok{0}\NormalTok{, }\DecValTok{20}\NormalTok{) +}
\StringTok{        }\KeywordTok{theme}\NormalTok{(}\DataTypeTok{legend.key =} \KeywordTok{element_rect}\NormalTok{(}\DataTypeTok{colour =} \OtherTok{NA}\NormalTok{))}
\KeywordTok{grid.arrange}\NormalTok{(ed.means.hist, ed.values.hist, }\DataTypeTok{ncol=}\DecValTok{2}\NormalTok{)}
\end{Highlighting}
\end{Shaded}

\end{document}
